% Please adhere to having exactly one clause or sentence per line of source.
% At the expense of 80-column purity, this makes diff output much more useful.

\documentclass[a4paper,12pt]{article}
\usepackage[utf8]{inputenc}
\usepackage[T1]{fontenc}
\usepackage{parskip}
\usepackage{hyperref}
\usepackage{graphicx}
\usepackage{enumerate}

\setcounter{secnumdepth}{3}

\hypersetup{
	pdfauthor=ComSSA,
	pdftitle=The Constitution of the Computer Science Students Association
}

\title{\scshape
	The Constitution\\
	of the\\
	Computer Science Students Association
}

\date{October 12, 2016}

\author{} % suppress 'LaTeX Warning: No \author given.'

\pagenumbering{gobble}
\thispdfpagelabel0

\raggedright

\begin{document}

\maketitle

\vspace{1in}

\begin{center}
	\includegraphics[width=3in]{logo/delan/comssalogo_crop_black.eps}
\end{center}

\newpage

\pagenumbering{arabic}

\section{Preliminary}

\subsection{Name}

The Club shall be known as the \textit{Computer Science Students Association}, also by the short name \textit{ComSSA} and hereafter referred to as ComSSA.

\subsection{Objective}

The objectives of this Constitution shall be:

\begin{enumerate}[a)]
	\item To provide a framework for the operation of the Club, and
	\item To assist the Members who operate the Club in a way that is consistent with the ethos of the Club.
\end{enumerate}

\subsection{Definition and Interpretation}

For the purposes of this Constitution, unless the contrary intention appears:

\textbf{``ComSSA''} is the short name for the Computer Science Students Association.

\textbf{``Club''} refers to ComSSA.

\textbf{``Constitution''} refers to this document.

\textbf{``Meeting''} refers to a meeting, other than an AGM or SGM, during which Committee Members may vote on motions which require a Committee vote, unless otherwise specified.

\textbf{``Committee Meeting''} refers to a ``Meeting''.

\textbf{``AGM''} stands for ``Annual General Meeting''.

\textbf{``SGM''} stands for ``Special General Meeting''.

\textbf{``Club Year''} refers to the period of time between the AGM of one year until the AGM of the next year.

\textbf{``Member''} refers to any member of ComSSA, unless otherwise qualified.

\textbf{``Committee''} refers to the group of Members who are responsible for the operation of the Club.

\textbf{``Committee Member''} refers to a person on the Committee.

\textbf{``Executive Committee''} refers to the subset of Committee Members who are also office bearers.

\textbf{``Executive Committee Member''} refers to a person on the Executive Committee.

\textbf{``OCM''} stands for ``Ordinary Committee Member''.

\textbf{``Club Asset''} refers to any item belonging to ComSSA, be it monetary, physical, abstract or otherwise.

\textbf{``Majority''} is defined as the nearest whole number to half of a cardinality, that is strictly greater than half of that cardinality.

\textbf{``Simple Majority''} is defined as a Majority of voting members present at the the meeting during which a given vote is taking place.

\textbf{``Absolute Majority''} is defined as a Majority of voting members at the current time, regardless of the number attending the meeting.

\textbf{``Guild'} means the Student Guild of Curtin University of Technology.

\textbf{``Guild Executive Committee'} means the Executive Committee of the Guild.

\textbf{``Guild President'} means the person for the time being holding the office or acting as President of the Guild.

\textbf{``Guild Statute Book'} means the Statute Book created under R1301 of the Guild Regulations.

\textbf{``Curtin Student Guild Full Member''} is defined as a full member of the Curtin Student Guild, as defined by the Curtin Student Guild Rules and Regulations.

For the purposes of this constitution the following interpretations shall apply:

Where in this Constitution the word \textbf{``may'} is used in conferring a function, it is to be interpreted to imply that the function so conferred can be exercised or not at discretion. Where in this Constitution the word \textbf{``shall'} is used in conferring a function it is to be interpreted to mean that the function conferred must be exercised.

\subsection{Registration}

ComSSA must register as a Curtin Student Guild-affiliated club each year, by whatever means set by the Curtin Student Guild Council for the current calendar year.

\subsection{Limit of the ComSSA Constitution}

This Constitution is restricted and subject to all current Curtin University Statutes, as well as all Curtin Student Guild rules, regulations, and policy pertaining to the operation of Curtin Student Guild-affiliated clubs.

\section{Objectives of the Club}

The objectives of the Club are:

\begin{enumerate}[a)]
	\item To provide social events for students and staff of the Department of Computing,
	\item To provide support for students entering university life,
	\item To provide support for students who need representation with the Department of Computing, and
	\item To liaise with the Department of Computing and function as a forum for student and staff relations.
	\item To become and remain registered with the Guild.
	\item To do all things that would appear necessary and proper for the benefit and advancement of the Society and the Guild.	
\end{enumerate}

\section{Powers}

The Society for the purpose of achieving its objectives shall have the power:

\begin{enumerate}[a)]
	\item To purchase, sell, lease or rent Society or personal property providing that this is in
accordance with Guild Statute Book.
	\item To borrow, raise or secure the payment of money to secure the payment of
performance of any debt, liability, contract or guarantee incurred or to be entered into
by the Society, provided that this is in accordance with Guild Statute Book.
	\item To exercise the rights and privileges associated with the registration of a Guild
Society.
	\item To invest the monies of the Society.
	\item To do all such other things as are incidental or conducive to the objects of the Society.
\end{enumerate}

\section{Membership}

\begin{enumerate}[a)]
	\item Membership of the Society shall consist of Ordinary Members and Associate Members.
	\item Ordinary Membership shall be open to all enrolled students of Curtin University who pay the membership fee as set by the Committee.
	\item  Associate Membership may be granted to any persons not eligible for Ordinary Membership at the discretion of the committee.
	\item At least 50\%+1 of the Ordinary Members of the Society shall be members of the Guild.
	\item Associate Members shall enjoy all the benefits of Ordinary Membership except that they shall not be eligible to:
	\begin{enumerate}
		\item Vote at any General Meeting of the Society;
		\item Be an Officer of the Society; or
		\item Directly benefit from any money received from the Guild.
	\end{enumerate}
\end{enumerate}

\subsection{Limits of Membership}

Membership of the Club is unrestricted excepting a unanimous vote by the Committee. 

\subsection{Membership Length}

All memberships shall remain active from full payment of the membership fee until the beginning of the first day of Teaching Week 1 of Semester 1 of the following academic year, as set forth in the Curtin University academic calendar for that year.

\subsection{Cancellation of Membership}

The membership of any Member may be cancelled by the Committee upon any of the following:

\begin{enumerate}[a)]
	\item A unanimous vote by the Committee, or
	\item Written agreement and support of the Curtin Student Guild Student Engagement Coordinator.
\end{enumerate}

Expulsion or refusal of membership on the grounds of race, age, gender or sexuality is prohibited. An appeal shall be presided over by the relevant Guild staff member or member of the Guild Executive.

\section{Fees and Membership Costs}

\subsection{Membership Fee}

Membership prices shall be voted upon at the first Committee Meeting of each calendar year, requiring an Absolute Majority to pass.

Should an increase be proposed, the proposed membership price increase may not exceed 50\% of the previous year's membership fees unless decided upon by a Simple Majority vote at a Special General Meeting.

All Curtin Student Guild Full Members shall receive at least 10\% off the membership price that they would've paid had they not been Curtin Student Guild Full Members.

\subsection{Compulsory Fees}

There shall be no compulsory fees charged to Members apart from the membership fee.

\subsection{Costs for Members to Attend ComSSA Events}

The maximum cost for Members to attend any official ComSSA event shall be fifteen (15) Australian dollars or 20\% above cost price, whichever is greater.

\section{Committee}

\subsection{Committee Structure}

\subsubsection{Office Bearers}
\label{office_bearers}

The Committee  - that shall be responsible for the administration of the Society - shall consist of the following voting office bearers:

\begin{enumerate}[a)]
	\item President,
	\item Vice President,
	\item Treasurer, and
	\item Secretary.
\end{enumerate}

\subsubsection{Ordinary Committee Members}

The Committee shall consist of the following voting non-office bearers:

A selection from zero (0) to seven (7) Ordinary Committee Members, inclusive.

\subsection{Length of Term}

Positions shall be held for one year; from the last day of the second semester (including any exam weeks) to the same time in the following year, as defined by the Curtin University academic calendar.

\subsection{Committee Eligibility Requirements}
\label{committee_eligibility}

To be considered for any committee position, a Nominee must meet all of the following criteria:

\begin{enumerate}[a)]
	\item The Nominee must be a current enrolled student at Curtin University, studying at the Bentley campus, and
	\item The Nominee must be a ComSSA member, and
	\item The Nominee must be an Ordinary Member of the Society who is a Guild Member.
\end{enumerate}

\subsection{Executive Committee Eligibility Requirements}
\label{exec_committee_eligibility}

To be considered for an office bearer position, a Nominee must meet all of the following criteria:

\begin{enumerate}[a)]
	\item The Nominee must be a Curtin Student Guild member, and
	\item At least one of the following must be true for the Nominee:
	\begin{enumerate}[i)]
		\item The Nominee must be studying at least one course which has greater than 50\% of its credit points assigned to units run by the Department of Computing, and/or
		\item The Nominee must be studying at least one course which will yield a degree which is deemed as ``relevant to computing'' by an academic staff member of the Department of Computing in the position of Senior Lecturer or higher.
	\end{enumerate}
\end{enumerate}

\subsection{Ordinary Committee Members}

The method of selection for Ordinary Committee Member positions --- as well as the time at which they will be selected --- shall be determined by the Executive Committee by means of an Absolute Majority vote at a Committee Meeting, SGM, or AGM.

\subsection{Quorum}

Quorum for any Committee Meeting is satisfied when the following conditions are met:

\begin{enumerate}[a)]
	\item Three (3) or more Executive Members must be present, and
	\item In the instance of:
	\begin{enumerate}[i)]
		\item Zero (0), one (1), or two (2) Ordinary Committee Members currently selected, there are no more requirements.
		\item Three (3) or more Ordinary Committee Members currently selected, five (5) total Committee Members must be present.
	\end{enumerate}
\end{enumerate}

A Committee Member may qualify to be counted in both of these conditions.

\subsection{Vacancy of Position}

Any position will become vacant upon any of the following criteria:

\begin{enumerate}[a)]
	\item The current occupant's written notice to the Committee,
	\item Absence from two (2) consecutive meetings without given notice, upon verification of the Committee via an Absolute Majority vote,
	\item Absence from four (4) consecutive meetings, upon verification of the Committee via an Absolute Majority vote, or
	\item No longer being a Member of ComSSA.
	\item No longer an enrolled student of Curtin University.
\end{enumerate}

\subsection{Vacancy or Incapacitation}

\begin{enumerate}[a)]
	\item The powers and responsibilities associated with all vacant positions shall immediately be absorbed by the remaining Committee Members.
	\item If the remaining Committee wishes to exercise any powers or responsibilities associated with any vacant position, it must do so via an Absolute Majority vote at a Committee Meeting.
	\item If the Committee chooses, it may --- via an Absolute Majority vote at a Committee Meeting --- select one person to absorb all powers and responsibilities associated with the vacated position. This person must meet the Committee Eligibility Requirements in \S~\ref{committee_eligibility}. If the position being filled is on the Executive Committee, this person must also meet the Executive Committee Eligibility Requirements in \S~\ref{exec_committee_eligibility}.
	\item A SGM must be held within minimum 14 days and a maximum of 28 days of resignation of an Executive Committee Member
\end{enumerate}

\section{Powers and Responsibilities}

\subsection{Duties of the Committee}

In addition to any powers and responsibilities outlined elsewhere in the Constitution, the Committee has the power and responsibility to:

\begin{enumerate}[a)]
	\item Plan activities in accordance with the Club objectives and inform all Members of these activities,
	\item Formulate Policy in accordance with the Club objectives,
	\item Strive to consistently have available committee members on campus as often as possible,
	\item Act according to all enacted Policy,
	\item Raise and spend funds in accordance with the current Spending Policy,
	\item Act in the best interests of the Club, and
	\item Operate the Club in an ethical manner.
	\item Expend monies for certain matters and to incur debts and liability on behalf of the Society for which the committee shall be liable.
\end{enumerate}

\subsection{Duties of the President}

The President has the power and responsibility to:

\begin{enumerate}[a)]
	\item Represent the Club in any matters relating to ComSSA,
	\item Co-ordinate and supervise the Committee,
	\item Familiarise the Members with the objectives of the club, and
	\item Call for Special General Meetings.
\end{enumerate}

\subsection{Duties of the Vice President}

The Vice President has the power and responsibility to:

\begin{enumerate}[a)]
	\item Assist the President in carrying out their responsibilities.
\end{enumerate}

\subsection{Duties of the Treasurer}

The Treasurer has the power and responsibility to:

\begin{enumerate}[a)]
	\item Keep a record of Club property and finances,
	\item Prepare a report of income and expenditure during the Club Year to be provided during an AGM or SGM,
	\item Write or make necessary amendments to the Spending Policy,
	\item Ensure the Committee's adherence to the Spending Policy,
	\item Familiarise and report to the Committee the financial status of the Club,
	\item Suspend a Committee Member upon a violation of the Spending Policy.
\end{enumerate}

\subsection{Duties of the Secretary}

The Secretary has the power and responsibility to:

\begin{enumerate}[a)]
	\item Record and keep minutes of all Committee meetings, and to ensure that those minutes are a complete and accurate record of a meeting's proceedings, as decided by a Simple Majority vote at the following meeting,
	\item Manage correspondence of the Club,
	\item Manage a list of current Members,
	\item Keep, publish and provide to any Member the Constitution, Policies and meeting minutes of the Club, and
\end{enumerate}

\subsection{Duties of Ordinary Committee Members}

The OCMs have the power and responsibility to:

\begin{enumerate}[a)]
	\item Vote as Committee Members,
	\item Perform tasks as delegated by the Committee,
	\item Assist the Committee in their duties, and
	\item Ensure the proper conduct of the Executive Committee Members.
\end{enumerate}

\subsection{Duties of the Members}

Members have the power and responsibility to:

\begin{enumerate}[a)]
	\item Vote during an AGM or SGM, and
	\item Become a part of the Committee.
\end{enumerate}

\section{Suspension and Removal of Committee Members}

\subsection{Suspension of Ordinary Committee Members}

\begin{enumerate}[a)]
	\item An Ordinary Committee Member may be placed under a state of suspension by two Executive Committee Members.
	\item Once suspended, the OCM loses all powers and responsibilities associated with their position, until one of the following conditions is met:
	\begin{enumerate}[i)]
		\item A period of fourteen (14) days passes since the suspension, or
		\item A Committee Meeting is held, during which the removal of the OCM in question is voted upon.
	\end{enumerate}
	\item Upon suspension of duty, the OCM must yield all Club Assets in their possession to the Executive Committee.
\end{enumerate}

\subsection{Removal of Ordinary Committee Members}

\begin{enumerate}[a)]
	\item Confirmation of suspension for an OCM suspended from active duty,
	\item A Majority vote by the full Committee, with the exception of the OCM being considered for removal:
	\begin{enumerate}[i)]
		\item The OCM in question may be removed from the meeting during discussion. This will be decided upon by a Simple Majority vote of present Committee Members excluding said OCM.
		\item At least two Committee Members may request that the OCM in question be removed from the meeting during the vote.
	\end{enumerate}
\end{enumerate}

\subsection{Removal of Executive Committee Members}

\subsubsection{Requests for Removal}

Any Member may request the removal of an Executive Committee Member from office. The request must be given in writing, and will only be considered if:

\begin{enumerate}[a)]
	\item The request includes reasons for removal from office, and
	\item The request includes the signatures of at least ten (10) Members or 10\% of the current membership, whichever is greater.
\end{enumerate}

\subsubsection{Consideration of Request}

A Committee Meeting must be held within fourteen (14) days of receipt of a request.

\begin{enumerate}[a)]
	\item The Members that requested removal must be notified of the time and place of the Committee Meeting.
	\item The Member(s) may present their reasons in person to the Committee.
	\item The Executive Committee Member in question may explain their actions to the Committee and the Member(s) in attendance.
	\item The Executive Committee Member will be removed from their position upon an Absolute Majority vote by the Committee, with the exception of the Executive Committee Member being considered for removal.
	\begin{enumerate}[i)]
		\item The Executive Committee Member in question may be removed from the meeting during discussion. This will be decided upon by a Simple Majority vote of present Committee Members excluding said Executive Committee Member.
		\item At least two Committee Members may request that the Executive Committee Member in question be removed from the meeting during the vote.
	\end{enumerate}
	\item The Committee must supply to the Executive Committee Member in question, and the raising Member(s), justification for the decision made.
\end{enumerate}

Upon removal from the Committee:

\begin{enumerate}[a)]
	\item The former Executive Committee Member must yield all Club Assets in their possession to the remaining Executive Committee Members.
	\item The Committee must supply to the former Executive Committee Member justification for removal.
	\item An SGM must be called within four (4) weeks of their removal.
\end{enumerate}

\section{Dismissal of Committee}

\subsection{Request for Dismissal}

Any Member may request for the dismissal of the entire Committee. The request must be given in writing, and will only be considered if:

\begin{enumerate}[a)]
	\item The request includes reasons for removal from office, and
	\item The request includes the signatures of at least twenty (20) members or 20\% of the current membership, whichever is greater.
\end{enumerate}

\subsection{Receipt of Request}

An SGM must be called within fourteen (14) days of receipt of request. The agenda shall include:

\begin{enumerate}[a)]
	\item Justification for dismissal of the Committee by the Member(s) making the request,
	\item The Committee explanation for the actions detailed in the justification, and
	\item A vote by the attending Members for dismissal of the entire Committee.
	\begin{enumerate}[i)]
		\item The method of voting for dismissal of the Committee shall be a secret ballot supervised by an independent source.
	\end{enumerate}
	\item Election of the new Committee (if necessary).
\end{enumerate}

\subsection{Confirmation of Dismissal}

Upon dismissal of the Committee:

\begin{enumerate}[a)]
	\item The dismissed Committee must yield all Club Assets in their possession to the newly elected Committee.
\end{enumerate}

\subsection{Right to Appeal}

The dismissed Committee may stand for re-election at the SGM, and as such has no right to appeal.

\section{Committee Meetings}

\subsection{Time to be Held}

There should not be a gap of more than 17 days between each Committee Meeting.

\subsection{The Chairperson}

\subsubsection{Chairperson Powers and Responsibilities}

Each Committee Meeting shall have a Chairperson. The Chairperson has the power and responsibility to:

\begin{enumerate}
	\item Direct the Committee Meeting according to the Agenda.
	\item Ensure that all Members are heard when appropriate.
	\item Silence a Member when they are interrupting discussion by means that the Chairperson sees as inappropriate.
\end{enumerate}

\subsubsection{Chairperson Selection}

\begin{enumerate}
	\item If the President is attending the Committee Meeting, the President is --- by default --- the Chairperson.
	\item If the President is not attending a Committee Meeting, the Vice President acts as Chairperson.
\end{enumerate}

\subsubsection{Dissent in the Chairperson}

At any point in time, the Committee may --- via a Simple Majority vote --- note dissent in the Chairperson. The Committee may then --- via a Simple Majority vote --- elect a new Chairperson for the duration of the Committee Meeting.

\subsection{Quorum}

Quorum is satisfied when the following conditions are met:

\begin{enumerate}[a)]
	\item Three (3) or more Executive Committee Members must be present, and
	\item In the instance of a:
	\begin{enumerate}[i)]
		\item Zero (0) Ordinary Committee Member configuration, there are no more requirements.
		\item Three (3), five (5), or seven (7) Ordinary Committee Member configuration, five (5) total Committee Members must be present.
	\end{enumerate}
\end{enumerate}

\subsection{Agendas}

\subsubsection{Purpose of Agendas}

The purpose of an Agenda is to:

\begin{enumerate}[a)]
	\item Provide insight as to the structure of a Committee Meeting before it happens,
	\item Provide guidance for the Committee Meeting's Chairperson, and
	\item Provide background information to Members on topics relating to planned discussion at the Committee Meeting.
\end{enumerate}

\subsubsection{Creation and Distribution of Agendas}

The President must:

\begin{enumerate}
	\item Create an Agenda for every Committee Meeting,
	\item Distribute the Agenda to all Committee Members no less than 48 hours before the Committee Meeting takes place,
	\item Allow any Member to add to the Agenda for an upcoming Committee Meeting if the matter they wish to have discussed relates to the Club's operation,
	\item Distribute the Agenda to any Members who wish to see it, and
	\item In the event that any Member indicates that they wish to be notified about the release of an upcoming Agenda, notify said Member once the Agenda has been distributed to Committee Members.
\end{enumerate}

\subsubsection{Standing Items}

The following items must appear in the agenda of all Committee Meetings:

\begin{enumerate}[a)]
	\item The time and place at which the Committee Meeting shall take place,
	\item A motion, approving the previous Committee Meeting's minutes as being a true and accurate portrayal of the proceedings of the last Committee Meeting,
	\item A motion, noting a report provided by the Treasurer outlining the Club's financial position, and
	\item A section for Members to raise issues and provide feedback regarding any aspect of the Club's operation.
\end{enumerate}

\section{Annual General Meetings}

\subsection{Time to be Held}

The AGM is to be held in the last four teaching weeks of Semester Two, as defined by the Curtin University academic calendar.

\subsection{Notice to be Given}

At least fourteen (14) days of notice is to be given to all Members detailing date, time and location of an AGM. The Guild shall be notified of any General Meeting of the Society.

\subsection{Agenda}

The agenda shall consist of, in order:

\begin{enumerate}[a)]
	\item A report from each Executive Committee Member,
	\item Ratification of the Constitution,
	\item Election of the Executive Committee, and
	\item Any matter that is in writing, signed by at least ten (10) Members or 10\% of the current membership, whichever is greater.
\end{enumerate}

\subsection{Election of Positions}

Any election that takes place shall be undertaken in line with \S~\ref{elections}.

\subsection{Quorum}

Quorum for the AGM shall be fifteen (15) Members or an Absolute Majority of the listed Members, whichever is fewer.

\subsection{Voting}

\begin{enumerate}[a)]
	\item The method of voting, with the exception of election of positions, shall be decided by the President, or in their absence, the Vice President.
	\item Any matter that is voted upon shall be deemed to be passed if a Majority of attending votes are in favour.
	\item Only Ordinary Members may vote
\end{enumerate}

\section{Special General Meetings}

\subsection{Convening}

An SGM can be called for by any of the following means:

\begin{enumerate}[a)]
	\item Written notice by an Absolute Majority of the Committee,
	\item Written notice from the President,
	\item Written notice by at least ten (10) Members or 10\% of the current membership, whichever is greater, or
	\item Written notice from the Guild.
\end{enumerate}

\subsection{Notice to be Given}

At least seven (7) days but no more than four (4) weeks of notice is to be given to all Members detailing date, time and location of an SGM. The Guild shall be notified of any General Meeting of the Society.

\subsection{Agenda}

The agenda shall consist of, in order:

\begin{enumerate}[a)]
	\item A report from each Executive Committee Member,
	\item Election of any vacant office bearer positions,
	\item Any matter in writing signed by at least ten (10) Members or 10\% of the current membership, whichever is greater, and
	\item Any matter in writing signed by an Absolute Majority of the Committee.
\end{enumerate}

\subsection{Quorum}

Quorum for an SGM shall be fifteen (15) Members or an Absolute Majority of the current membership, whichever is fewer.

\subsection{Election of Positions}

Any election that takes place shall be undertaken in line with \S~\ref{elections}.

\subsection{Voting}

\begin{enumerate}[a)]
	\item The method of voting, with the exception of election of positions and dismissal of the Committee, shall be decided by the President, or in their absence, the Vice President.
	\item Any matter that is voted upon shall be deemed to be passed if a Majority of attending votes are in favour.
	\item Only Ordinary Members may vote at a General Meeting.
\end{enumerate}

\section{Elections}
\label{elections}

\subsection{Method of Election}

\begin{enumerate}
	\item A voter is required to indicate a preference for each candidate on the ballot paper by using the numbers 1, 2, 3 etc. up to the number of candidates.
	\item A candidate must poll an Absolute Majority of all formal votes to be elected.
	\item If, after all first preference votes have been counted, no candidate has obtained an absolute majority of all formal votes, then the candidate with the fewest number of first preference votes is excluded. That excluded candidate's second preference votes are then distributed to the remaining candidates.
	\item If after that exclusion no candidate has obtained an absolute majority of formal votes, the next remaining candidate with the fewest votes is excluded and ALL of his/her votes (i.e. first preference votes PLUS those votes received from the first excluded candidate) are distributed to the remaining candidates.
	\item The above process is continued until one candidate obtains an absolute majority of formal votes and is elected.
	\item If at any exclusion, the next available preference is for a previously excluded candidate, then that preference is disregarded and the vote is distributed to the continuing candidate for whom the next available preference is shown.
\end{enumerate}

\subsection{The Returning Officer}

\begin{enumerate}[a)]
	\item All Elections shall have a Returning Officer, who shall be completely impartial to all aspects of the election.
	\item The Returning Officer must be approved via a Simple Majority vote by all voters.
	\item All vote-counting shall be done by the Returning Officer.
	\item The Returning Officer shall allow any voter to witness the vote count.
\end{enumerate}

\section{Policies}

\subsection{Definition}

A Policy is a formal, standardised document, used to record a rule set or process related to the running of the Club.

\subsection{Forming Policy}

Except where otherwise stated in this document, any ComSSA Member (or group of Members, including any and all Committee Members) may form a Policy to be considered by the Committee.

\subsection{Consideration Conditions}

In order for a Policy to be considered by the Committee, it must meet the Consideration Conditions:

\begin{enumerate}[a)]
	\item Be in writing,
	\item State the full name of the person(s) that contributed to the Policy, hereby referred to as the Submitter(s),
	\item Be endorsed by one (1) Committee Member; this may be one of the Submitter(s), and
	\item State the full name of the Committee Member that endorsed the Policy, hereby referred to as the Endorser.
\end{enumerate}

\subsection{Considering and Enacting a Policy}

When a Policy meets the Consideration Conditions, it may be considered by the Committee.

\begin{enumerate}[a)]
	\item In order for a Policy to be voted upon in a meeting, it must be included in the agenda for said meeting.
	\begin{enumerate}[i)]
		\item The Policy document must be included along with the agenda upon circulation to Committee Members. The intent of this condition is to allow Committee Members to process a Policy and its possible ramifications before voting.
	\end{enumerate}
	\item At the Committee meeting (hereafter the Policy Meeting) the Policy will undergo an initial out-loud reading in its entirety by the chairperson.
	\item The Policy will be discussed by the Committee Members present at the Policy Meeting.
	\item Before voting takes place, amendments may be made to the Policy by the Submitter(s).
	\begin{enumerate}[i)]
		\item If any amendments are made to a Policy, they must be approved by all Submitter(s).
		\item At any point during the Policy Meeting, any of the Submitter(s) may remove their name(s) from the Policy.
		\item At any point during the Policy Meeting, and with the permission of all other Submitter(s) (or with nobody's permission if the Policy is has been left without any Submitters), someone may add themselves to the list of Submitter(s) and make amendments.
		\item If the Policy is without Submitter(s), with nobody willing to become a Submitter, discussion is dropped.
		\item At any time during discussion, the Endorser may cease endorsing the Policy. If no Committee Members endorse the Policy, discussion is dropped.
	\end{enumerate}
	\item Once all Submitter(s) and the Endorser are satisfied with the state of the Policy, the Policy will go to a vote by the Committee Members present at the Policy Meeting. If a Committee member is not present at the Policy Meeting, they may choose to vote remotely as an absentee.
	\begin{enumerate}[i)]
		\item Any two (2) Committee Members attending the Policy Meeting may request to have voting postponed until the next Committee meeting (which becomes the Policy Meeting upon reenactment of this process).
		\item If said Committee member(s) are not present at the next Policy Meeting, they forfeit their right to vote on the Policy.
		\item A Policy Meeting may not be postponed more than once per Policy.
	\end{enumerate}
	\item Policy is considered Enacted when it receives an Absolute Majority vote by the Committee.
\end{enumerate}

\subsection{Nullification of a Policy}

A Policy can be nullified by the Committee via an Absolute Majority vote at a Committee meeting.

\subsection{Amendment Consideration Conditions}

A Policy may be amended after enactment.

In order for a Policy amendment to be considered by the Committee, it must meet the Consideration Conditions:

\begin{enumerate}[a)]
	\item Be in writing,
	\item Be based on an existing enacted Policy,
	\item State the full name of the person(s) that contributed to the Policy amendment, hereby referred to as the Amender(s),
	\item Be endorsed by one (1) Committee member; this may be one of the Amender(s), and
	\item State the full name of the Committee member that endorsed the Policy, hereby referred to as the Endorser.
\end{enumerate}

\subsection{Amendment of a Policy}

The process for amending a Policy is as follows:

For the duration of the following procedure, Amender(s) refers to the Amender(s) for the current amendment only.

\begin{enumerate}[1)]
	\item In order for a Policy amendment to be voted upon in a meeting, it must be included in the agenda for said meeting.
	\begin{enumerate}[i)]
		\item The amended Policy document must be included along with the agenda upon circulation to Committee Members.
		\item Inherently, a Policy amendment may not be voted upon with less than 48 hours notice to all Committee Members.
	\end{enumerate}
	\item At the next Committee meeting (hereafter the Policy Meeting) the amended Policy will undergo an initial out-loud reading in its entirety by the chairperson.
	\item The amended Policy will be discussed by the Committee Members present at the Policy Meeting.
	\item Before voting takes place, further amendments may be made to the Policy by the Amender(s).
	\begin{enumerate}[i)]
		\item If any further amendments are made to a Policy, they must be approved by all Amender(s).
		\item At any point during the Policy Meeting, any of the Amender(s) may remove their name(s) from the amended Policy.
		\item At any point during the Policy Meeting, and with the permission of all other Amender(s) (or with nobody's permission if the Policy is has been left without a submitter), someone may add themselves to the Amender(s) list and make further amendments.
		\item If the Policy is without Amender(s), with nobody willing to become a member of Amender(s), discussion is dropped and the Policy is left as-is.
		\item At any time during discussion, the Endorser may cease endorsing the Policy amendment. If no Committee Members endorse the Policy amendment, discussion is dropped.
	\end{enumerate}
	\item Once all Submitter(s) and the Endorser are satisfied with the new state of the Policy, The new Policy will go to a vote by the Committee Members present at the Policy Meeting. If a Committee member is not present at the Policy Meeting, they forfeit their right to vote on said Policy amendment.
	\begin{enumerate}[i)]
		\item Any two (2) Committee Members attending the Policy Meeting may request to have voting postponed until the next Committee meeting, which becomes the Policy Meeting.
		\item If said Committee member(s) are not present at the next Policy Meeting, they forfeit their right to vote on the Policy.
		\item A Policy Meeting may not be postponed more than once per Policy amendment.
	\end{enumerate}
	\item The amended Policy is considered Enacted when it receives an Absolute Majority vote by the Committee. The now-second-newest Amendment of the Policy will no longer be considered as Enacted.
\end{enumerate}

\subsection{Record of Policies}

The Secretary must keep a record of current, nullified, and rejected Policies, as well as Policies undergoing discussion by the Committee.

This record must include:

\begin{enumerate}[a)]
	\item An accurate and factual record of all Policies, as voted on by the Committee.
	\item The Submitter(s) and Endorser responsible for each Policy.
	\item All amendments (past and present) of any Policies that have been amended.
	\item An objective summary of the changes that took place with each amendment.
	\item The Amenders and Endorser responsible for each amendment.
\end{enumerate}

\subsection{Restriction of Policy}

All Policy is restricted by certain terms, and is to be held below the Constitution. Policy may not:

\begin{enumerate}[a)]
	\item Contradict or nullify anything set forth in the Constitution, or
	\item Contradict or nullify anything set forth in any rules or regulations the Constitution is restricted by. See the section ``Limit of the ComSSA Constitution''.
\end{enumerate}

\subsection{Mandatory Policies}

In order to ensure proper operation of the Club, certain Policies are required.

\subsubsection{Spending Policy}

The Spending Policy defines and regulates the use of funds by the Committee.

The Treasurer must appear as a Submitter on the initial Spending Policy, and an Amender on any amendments.

The Spending Policy must enact the following clauses:
\begin{enumerate}[a)]
	\item The assets and income of the Club shall be applied solely in furtherance of its objectives as outlined in the Constitution, and no portion shall be distributed directly or indirectly to the Members of the Club except as bona fide compensation for services rendered or expenses incurred on behalf of the Club.
	\item In the event of the Club being dissolved, the amount that remains after such dissolution and the satisfaction of all debts and liabilities shall be transferred to the Curtin Student Guild.
\end{enumerate}

\section{Finances and Records}

\begin{enumerate}[a)]
	\item The Committee shall ensure true accounts are kept of the monies received and expended.
	\item A balance sheet containing a summary of assets and liabilities of the Society together with a statement of income and expenditure for the preceding year shall be made out and submitted
to the next Annual General Meeting.
	\item The Society shall inform the Guild of any bank accounts it holds and the signatories of that
account.
	\item The authority to access accounts shall rest with the at least two (2) Officers of the Society.
	\item The signatories of the accounts shall be any two (2) of the Officers of the Society.
\item The accounts shall be open to inspection by the Guild or a member of the Society upon giving
reasonable notice to the Treasurer at a time and place at the relevant Curtin campus
convenient to the Treasurer.
	\item The income and property of the Society shall be applied solely towards the promotion of the
objects of the Society.
	\item No portion of the income or property shall be paid, transferred or distributed directly or indirectly to the members of the Society, provided that nothing shall prevent the payment in good faith or remuneration in return for services actually rendered to the Society.
	\item The Society shall not sell any property without the consent of the Guild.
	\item Minutes of all meetings shall be open to inspection by any member or the Guild at a time and
place convenient to the Secretary.
	\item The Constitution shall be open to inspection by the Guild or any member upon giving
reasonable notice to the Secretary at a time and place at the relevant Curtin campus
convenient to the Secretary.
	\item If upon the dissolution or winding up of the Society there remains any properties whatsoever
after the satisfaction of all debts and liabilities, the same shall not be distributed among the members of the Society, but shall be given or transferred to the Guild.
\end{enumerate}

\section{Miscellaneous}

\begin{enumerate}[a)]
	\item The Society shall be affiliated as a Student Society with the Student Guild.
	\begin{enumerate}
	\item In the event of the dissolution of the Society, the assets, accounts and records of the Society shall be turned over to Guild or at the Guild’s approval shall be turned over to
	any Society of similar objectives to the dissolved entity.
	\item The Society shall not seek to loan, to give loans, to borrow or to enter into any form of
	written agreement with any person, body or organisation without the prior consent of
	the Guild.
	\item The Society shall meet any requirements made by the Guild Council.
	\item The Society shall inform the Guild President immediately in writing of any liabilities or
	possible liabilities that it may incur that may in any way become the ultimate
	responsibility of the Guild.
	\item The Society shall seek approval from the Guild Council for any intending action of the
	Society that may result in the possible liability of the Guild.
	\item The Society is bound by all the Rules, Regulations and other provisions specified by
	the Guild.
	\item Failure of the Society or its representatives to fulfil the obligations of the society as
	outline in the clauses above shall result in the Guild and its representatives being indemnified by the affiliate against any action, liability or possible consequences that may occur.
	\item The Society shall always represent Curtin University and the Guild in a positive light.
	\item The Society shall abide by all Guild and Curtin University Policies.
	\item The Society or an Officer of the Society may be referred and be subject to the
	Discipline Committee of the Guild for acts of misconduct.
	k. In accordance with Statute No.4 Student Guild the Guild Council may assume the
	care, control and management of the property and financial affairs of any student
	society, for such period as the Guild Council thinks fit; 
	\end{enumerate}
\end{enumerate}

\section{This Constitution}

\begin{enumerate}[a)]
	\item This Constitution shall be subject to the Guild Statute Book and where an inconsistency occurs the Statute Book shall prevail.
	\item Where a dispute arises regarding a matter under this constitution it may be referred to the Guild Executive Committee for resolution. The decision of the Guild Executive Committee shall be final.
	\item No alteration, addition or amendment of this Constitution shall be made unless and until carried by a resolution at any General Meeting called for such purpose by a majority of three fourths (75\%) of the members present.
	\item No such amendment shall have any force until the proposed change has been approved by the Guild Executive Committee.
	\item Notice of any proposed alteration, addition or amendment shall be given with seven (7) days notice to all members.
	\item The Society may, at any time, with the consent of the majority of three fourths (75\%) of the members present at a General Meeting called for the purpose, be dissolved.
	\item Such dissolution is to be notified to the Guild in writing.
	\item This Constitution was accepted by three fourths (75\%) or greater of the voting members present at the General Meeting, as per the information and details as listed on the Student Society Registration Form.
\end{enumerate}

\end{document}
